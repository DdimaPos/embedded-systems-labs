\section{Domain Analysis}
\subsection{Application Context and Utilized Technologies}
The primary objective of this laboratory work is to establish a communication bridge between a workstation (PC) and 
an embedded system (MCU) using the \textbf{UART (Universal Asynchronous Receiver-Transmitter)} protocol. 
This interaction is fundamental in embedded engineering for debugging, real-time sensor monitoring, 
and remote command execution.

The application uses the standard input/output library (\texttt{stdio.h}) adapted for microcontroller environments. 
This approach allows the use of classic C functions, such as \texttt{printf()} and \texttt{scanf()/fgets()}, to abstract 
the hardware registers of the serial port. This provides a human-readable interface for transmitting and receiving 
text-based commands, moving away from raw byte manipulation to a more sophisticated command-line interaction.

\subsection{Hardware and Software Components}

\begin{center}
\begin{tabular}{| p{4cm} | p{10cm} |}
\hline
\textbf{Component} & \textbf{Description \& Role} \\ \hline
\textbf{LAFVIN R3 (Arduino UNO R3)} & Microcontroller board that contains the central processing unit (ATmega328P) that parses the incoming character stream and toggles GPIO pin states. \\ \hline
\textbf{Serial-to-USB Interface} & Facilitates the translation of the microcontrollers's logic levels into USB protocol for communication with the terminal. \\ \hline
\textbf{LED - light emiting diode}  & The output peripheral used for visual validation. It uses a semiconductor that emits light at $20 mA$ when electrical current flows through it \\ \hline
\textbf{$220\,\Omega$ Resistor} & The resistor ensures current limiting to protect the component. Overwise the LED will burn out \\ \hline
\textbf{Neovim + Arduino-nvim plugin} & The chosen development environment (Text editor + plugin, arduino-cli). Neovim provides high-efficiency editing, while the plugin integrates \texttt{arduino-cli} for compilation. \\ \hline
\textbf{arduino-cli} & \texttt{arduino-cli} is a more advanced cli tool for interacting with MCU. Useful for automation and usually provides faster workflow that Arduino IDE or VSCode plugins. \\ \hline
\textbf{Breadboard} & A construction base that contains internal conductive strips to create electrical paths. Used for prototyping electronics that allows components to be interconnected without soldering. \\ \hline
\textbf{Serial Monitor} & The terminal emulator (Ghostty) with \texttt{arduino-cli monitor} command used to send the "led on" and "led off" strings and display confirmation responses. Role of the interface for sending signals to MCU \\ \hline
\end{tabular}
\end{center}

\subsection{System Architecture and Solution Justification}
The system adopts a \textbf{Command Parser architecture}. This solution was chosen to ensure extensibility; 
by utilizing the \texttt{STDIO} library, the code becomes more portable and readable compared 
to direct manipulation of UART circular buffers using the device-specific libraries.

The data flow is structured as follows:
\begin{enumerate}
    \item \textbf{Input:} The user enters a string command via the Ghostty terminal or another terminal emulator.
    \item \textbf{Transmission:} Data is sent asynchronously to the MCU via the TX/RX lines.
    \item \textbf{Processing:} Functions like \texttt{fgets()}  capture the string, which is then passed to 
      a logic module that compares the input against predefined instructions.
    \item \textbf{Action:} The MCU interacts with the LED driver module to change the physical state of the hardware.
\end{enumerate}

\subsection{Relevant Case Study}
A real-world application of this concept is the \textbf{Command Line Interface (CLI)} found in network infrastructure 
equipment. Technicians connect via a console port (serial) to input text commands for hardware configuration. 
Similarly, in industrial automation, this interface is used for calibration of sensors where a complex 
graphical user interface (GUI) is not possible to get or required. In this case specialists may need to 
rely on visual outputs (led blinking patterns) or outputs in terminal interface
