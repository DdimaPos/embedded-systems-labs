\subsection{Results}

For running the application following commands were used for compiling, 
uploading the project and monitoring the port to send the commands.

\begin{verbatim}
# to find to which port number the board is connected
arduino-cli board list 
# compilation of the programm by indicating the target hardware
arduino-cli compile --fqbn arduino:avr:uno sketch
# uploading the program executables to the board
arduino-cli upload -p /dev/cu.usbserial-portnumber --fqbn arduino:avr:uno sketch
# monitoring the port to which the active board is connected
arduino-cli monitor -p /dev/cu.usbserial-portnumber 
\end{verbatim}

Below is the output of the commands listed above

\begin{verbatim}
$ arduino-cli compile --fqbn arduino:avr:uno sketch              
Sketch uses 3918 bytes (12%) of program storage space. Maximum is 32256 bytes.
Global variables use 400 bytes (19%) of dynamic memory, leaving 1648 bytes for local variables. Maximum is 2048 bytes.

$ arduino-cli upload -p /dev/cu.usbserial-1120 --fqbn arduino:avr:uno sketch                                         
New upload port: /dev/cu.usbserial-1120 (serial)

$ arduino-cli monitor -p /dev/cu.usbserial-1120                                                                     
Using generic monitor configuration.
WARNING: Your board may require different settings to work!

Monitor port settings:
  baudrate=9600
  bits=8
  dtr=on
  parity=none
  rts=on
  stop_bits=1

Connecting to /dev/cu.usbserial-1120. Press CTRL-C to exit.

[DEBUG] System Ready. type led 
[DEBUG] System Ready. type led on/led off to control the led
led on
[INFO] Led was turned on
led off
[INFO] Led was turned off
led sos
[SOS] SOS SOS
\end{verbatim}

Below there is the visual representation of the circuit

\begin{figure}[h]
    \centering
    \includegraphics[width=0.6\textwidth]{img/ledoff.jpg}
    \caption{Led state after typing "led off"}
    \label{fig:ledoff}
\end{figure}

After typing \texttt{led off} the visual state of the 
LED does not change (Figure \ref{fig:ledoff}). This is expected behavior since the electrical
in not emitter from the LED pin.

\clearpage
\begin{figure}[h]
    \centering
    \includegraphics[width=0.6\textwidth]{img/ledon.jpg}
    \caption{Led state after typing "led on"}
    \label{fig:ledon}
\end{figure}

After typing \texttt{led on} the visual state of the 
led changes to active, since now the electrical current is emited
on the pin 13 (Figure \ref{fig:ledon}).

Additionally the \texttt{led sos} command was implemented by using the
\texttt{delay()} function that will stops the execution by specified time
which allows to send the signals at the desired intervals of time. In this case it is used to
send the sos signal in Morse code \texttt{...\_\_\_...} (3 long signals, 3 short signals, 3 long signals). 
The result of this command and other commands can be seen in the video posted 
on YouTube video hosting platform. \url{https://youtu.be/vp0BN4FRQHY?si=EbB3ROkQ5iNmlyLn}

