\section{Design}

\subsection{Architectural sketch}

Below is the architectural schetch of the components structured in the diagram that highlights the 
software and hardware components.

\begin{figure}[h]
    \centering
    \includegraphics[width=0.6\textwidth]{./img/architecturediagram.jpg}
    \caption{Architectural sketch of the project}
    \label{fig:archi}
\end{figure}

This is the description of each of the elements of the diagram:
\begin{enumerate}
  \item \textbf{Arduino Uno microcontroller} - central processing unit board that serves as a executor
    of software
  \item \textbf{LED} - hardware component which represents a semiconductor diode that serves as output
    for the user
  \item \textbf{Serial STDIO module} - software module that serves as driver for adapting the default I/O 
    commands like \texttt{printf();scanf()} to work with MCU's serial write and read
  \item \textbf{Led module} - software module that is a driver that controls internally the state of the 
    LED and provides the user-friendly abstractions that will not disclose internals about how it is implemented.
\end{enumerate}

\subsection{Electrical sketch}

The circuit diagram outlines the setup for this project, showing the connections between
the Arduino, an LED, and the computer for serial communication. It details the LED’s
connection to the Arduino’s pin 13 (default LED pin), including a resistor to limit current and prevent damage
to the LED by keeping it within safe electrical limits. The LED’s circuit is completed by
connecting its ground to the Arduino’s ground.


\begin{figure}[h]
    \centering
    \includegraphics[width=0.4\textwidth]{./img/circuit_sketch_wokwi.jpg}
    \caption{Circuit sketched on wokwi}
    \label{fig:wokwi}
\end{figure}

Figure \ref{fig:wokwi} details the connections and components that make up the system. This visual
documentation is crucial for replicating the setup or diagnosing issues, as it clearly shows
how each component is integrated into the overall design. Understanding the electrical
connections is essential for both the construction and troubleshooting phases of the project.

\subsection{Flow chart}

The Figure \ref{fig:flow} highlights how the Arduino’s USB interface is used for serial com-
munication with a computer. This enables the use of the Arduino IDE or terminal software
to issue commands like ”led on” and ”led off” to the Arduino, which then switches the LED
on or off accordingly.

\begin{figure}[h]
    \centering
    \includegraphics[width=0.62\textwidth]{./img/flow.jpg}
    \caption{Flow chart of the algorithm}
    \label{fig:flow}
\end{figure}

\subsection{Code design}

Code is developed to followed the principle of the interface and the 
implementation, by creating for each header file \texttt{.h} a respective 
\texttt{.cpp} file that implements it. At the same time the most important
part of the implementation is separation of the \textbf{driver} and 
\textbf{abstraction} layer. Abstraction layer is the main portable program that calls
low level modules. This makes the program portable to any device, ensuring that only the driver
level is adapted properly. Driver level contains the low level logic related to specific
hardware. In my case it contains all the calls to Serial port and LED power control using 
the \texttt{<Arduino.h>} library.
