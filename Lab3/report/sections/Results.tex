\subsection{Results}

For running the application following commands were used for compiling,
uploading the project and monitoring the port to observe the periodic reports.

\begin{verbatim}
# to find to which port number the board is connected
arduino-cli board list
# compilation of the programm by indicating the target hardware
arduino-cli compile --fqbn arduino:avr:uno sketch
# uploading the program executables to the board
arduino-cli upload -p /dev/cu.usbserial-portnumber --fqbn arduino:avr:uno sketch
# monitoring the port to which the active board is connected
arduino-cli monitor -p /dev/cu.usbserial-portnumber
\end{verbatim}

Below is the output of the commands listed above

\begin{verbatim}
$ arduino-cli compile --fqbn arduino:avr:uno sketch
Sketch uses 3918 bytes (12%) of program storage space. Maximum is 32256 bytes.
Global variables use 400 bytes (19%) of dynamic memory, leaving 1648 bytes
for local variables. Maximum is 2048 bytes.

$ arduino-cli upload -p /dev/cu.usbserial-1120 --fqbn arduino:avr:uno sketch
New upload port: /dev/cu.usbserial-1120 (serial)

$ arduino-cli monitor -p /dev/cu.usbserial-1120
Using generic monitor configuration.

Monitor port settings:
  baudrate=9600

Connecting to /dev/cu.usbserial-1120. Press CTRL-C to exit.

Lab 3.1 - Button Press Monitor
Reports every 10 seconds
--- Report ---
Total presses: 5
Short (<500ms): 3
Long  (>=500ms): 2
Avg duration: 487 ms
--------------
--- Report ---
Total presses: 0
Short (<500ms): 0
Long  (>=500ms): 0
Avg duration: N/A
--------------
\end{verbatim}

The system behavior is as follows:
\begin{itemize}
  \item When a short press ($< 500$ ms) is detected, the \textbf{green LED} turns on and the
    red LED turns off. The \textbf{yellow LED} blinks 5 times rapidly.
  \item When a long press ($\geq 500$ ms) is detected, the \textbf{red LED} turns on and the
    green LED turns off. The \textbf{yellow LED} blinks 10 times rapidly.
  \item Every 10 seconds, a statistics report is printed to the serial monitor showing the
    total number of presses, the count of short and long presses, and the average press duration.
    After printing, all counters are reset.
  \item When no presses occur during a reporting interval, the report shows zeros and
    ``N/A'' for the average duration.
\end{itemize}

Below there is the visual representation of the circuit

\begin{figure}[h]
    \centering
    \includegraphics[width=0.6\textwidth]{img/ledoff.jpg}
    \caption{Circuit in idle state}
    \label{fig:ledoff}
\end{figure}

Figure \ref{fig:ledoff} shows the circuit in its idle state with all LEDs off.

\clearpage
\begin{figure}[h]
    \centering
    \includegraphics[width=0.6\textwidth]{img/ledon.jpg}
    \caption{LED active after a button press}
    \label{fig:ledon}
\end{figure}

Figure \ref{fig:ledon} shows the circuit after a button press has been detected,
with the corresponding LED (green or red) active to indicate the press duration category.
