\section{Conclusion}

Completion of this laboratory work has provided insightful understanding of
non-preemptive cooperative multitasking on bare-metal microcontroller systems
and the design principles behind real-time task scheduling.

This third laboratory work covered the implementation of a cooperative scheduler
with multiple tasks running at different recurrence intervals, communicating
through shared global variables. The key takeaways are:

\begin{enumerate}
  \item Designing a \textbf{cooperative scheduler} using context structures with
    recurrence, offset, and countdown counters -- ensuring only one task executes
    per system tick for deterministic behavior.
  \item Implementing \textbf{inter-task communication} through global variables
    and flags (\texttt{newPressDetected}, \texttt{reportReady}), demonstrating
    how tasks can exchange data without mutexes in a non-preemptive environment.
  \item Structuring the application into \textbf{three distinct layers}: driver,
    service (scheduler), and application -- maintaining portability and modularity.
  \item Understanding the importance of \textbf{separating ISR-context work from
    main-loop work}: the timer ISR drives the scheduler, while blocking operations
    like \texttt{printf()} are deferred to the main loop via a flag.
  \item Gaining practical experience with \textbf{button debouncing, press duration
    measurement}, and visual signaling through multiple LEDs at different rates.
\end{enumerate}

This laboratory work serves as a foundation for understanding operating system
concepts on embedded systems, bridging the gap between bare-metal sequential
programming and full RTOS-based multitasking.
